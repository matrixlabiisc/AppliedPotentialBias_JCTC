\dftfe{} is a C++ code for materials modeling from first principles using Kohn-Sham density functional theory.
It is based on adaptive finite-element discretization that handles all-electron and pseudopotential calculations in the 
same framework, and incorporates scalable and efficient solvers for the solution of the Kohn-Sham equations. Importantly, \dftfe{} 
can handle general geometries and boundary conditions, including periodic, semi-periodic and non-periodic systems. \dftfe{} code 
builds on top of the deal.II library for everything that has to do with finite elements, geometries, meshes, etc., and, through 
deal.II on p4est for parallel adaptive mesh handling.

\subsection{Authors}
\label{sec:authors}
\dftfe{} is developed and maintained by the \href{http://www-personal.umich.edu/~vikramg/}{Computational Materials Physics
group at University of Michigan} (directed by Prof. Vikram Gavini), and the \href{https://sites.google.com/view/matrix-lab}{MATRIX lab, Indian Institute of Science} (directed by Prof. Phani Motamarri). The code is maintained by a group of principal developers, 
who manage the architecture of the code and the core functionalities. Developers with
significant contributions to core functionalities and code architecture in the past who are 
no longer active principal developers, are listed under principal developers emeriti. 
A subset of the principal developers and mentors are administrators. Finally, all contributors who have
contributed to major parts of the DFT-FE code or sent important fixes and enhancements are listed
under contributors. All the underlying lists are in alphabetical order. 

\paragraph{Principal developers}
\begin{itemize}
	\item Sambit Das (University of Michigan, USA).
	\item Phani Motamarri (Indian Institute of Science, Bangalore, India).
\end{itemize}

\paragraph{Principal developers emeriti}
\begin{itemize}
        \item Phani Motamarri (University of Michigan, USA).
	\item Krishnendu Ghosh (University of Michigan, USA).
	\item Shiva Rudraraju (University of Wisconsin Madison, USA).	
\end{itemize}

\paragraph{Contributors}
\begin{itemize}
	\item Sambit Das (University of Michigan, USA).
	\item Vishal Subramanian (University of Michigan, USA).
	\item Bikash Kanungo (University of Michigan, USA).
 \item Ian C. Lin (University of Michigan, USA 2018-2019)
 \item Krishnendu Ghosh (University of Michigan, USA 2017-2020).
 \item Shiva Rudraraju (University of Michigan, USA 2015-2018)
 \item Shukan Parekh (University of Michigan, USA Summer 2019).
\item Phani Motamarri (Indian Institute of Science, Bangalore, India).
\item Nikhil Kodali (Indian Institute of Science, Bangalore India)
\item Kartick Ramakrishnan (Indian Institute of Science, Bangalore)
\item Gourab Panigrahi (Indian Institute of Science, Bangalore, India)
\item Srinibas Nandi (Indian Institute of Science, Bangalore, India)
\item David Rogers (Oak Ridge National Laboratory, USA)
\item Denis Davydov (University of Erlangen-Nuremberg, Germany). 	
	
\end{itemize}

\paragraph{Mentors/Development Leads}
\begin{itemize}
    \item Sambit Das (University of Michigan, USA)    
     \item Vikram Gavini (University of Michigan, USA).
     \item Phani Motamarri (Indian Institute of Science, Bangalore, India)
\end{itemize}

\subsection{Acknowledgments}
The development of \dftfe{} open source code relating to pseudopotential calculations has been funded in part by the Department 
of Energy PRISMS Software Center at University of Michigan, and the Toyota Research Institute. The development of \dftfe{} open 
source code relating to all-electron calculations has been funded by the Department of Energy Basic Energy Sciences. The methods 
and algorithms that have been implemented in \dftfe{} are outputs from research activities over many years that have been 
supported by the Army Research Office, Air Force Office of Scientific Research, Department of Energy Basic Energy Sciences, 
National Science Foundation and Toyota Research Institute. 

Part of the DFT-FE open source code development related to the capabilities --- PAW, Ab initio MD, NEB, non-collinear magnetism with spin-orbit coupling has been funded by seed grant from Indian Institute of Science; Science and Engineering Research Board (SERB) Startup Research Grant (SRG); National Supercomputing Mission (NSM) R\&D for Exascale Grant from the Department of Science and Technology (DST), India; Indo-Korean Science and Technology Center (IKST), Bangalore [South Korea] and Google India Research Award.

\subsection{Referencing \dftfe{}}
Please refer to \href{https://sites.google.com/umich.edu/dftfe/referencing}{referencing  \dftfe{}} to properly cite the use of 
\dftfe{} in your scientific work. 
